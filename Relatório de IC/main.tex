\documentclass[a4paper, 12pt]{article}
\usepackage[utf8]{inputenc}%Pacote para acentuação
\usepackage[brazil]{babel} % Escrita em português brasileiro.
\usepackage[lmargin=3cm,tmargin=3cm,rmargin=2cm,bmargin=2cm]{geometry} %Formato que lembra a ABNT
\usepackage[T1]{fontenc} %Ajusta o texto que vem de outras fontes
\usepackage{amsmath,amsthm,amsfonts,amssymb,dsfont,mathtools,blindtext} %pacotes matemáticos

\begin{document}
\maketitle

\section{Resumo}

\section{Palavras-chave}

\section{Introdução}

\section{Metodologia}

\subsection{Treinamento de programas que serão utilizados ao longo do projeto.}

 Podemos dizer que a metodologia utilizada nesta pesquisa se divide em algumas etapas sendo elas as de treinamento, implementação, teste e incorporação.
 Primeiramente começamos com a implementação do treinamento do programa linux através de reuniões semanais assíncronas, onde o professor ensina a desenvolver  comandos básicos como scripts, strings e bash que logo serão utilizados para compartilhamento de dados da pesquisa.
  Em seguida começando o plano de estudo dirigido de python, onde através de arquivos de diretórios python, e com ajuda da disciplina de Lógica de Programação, pude desenvolver algumas habilidades de operadores aritméticos, módulos, funções, condições e estruturas de repetição.
  Também tivemos uma breve introdução de supercomputadores através de um encontro assíncrono, compreendemos um pouco sobre cluster’s (supercomputadores) onde vimos um pouco de Machine Learning através do NPAD.

\subsection{Implementação do algoritmo em códigos.}

 Após a fase de treinamento, começamos a desenvolver a implementação de um código de algoritmo orientado a objeto, no qual introduzimos a \textbf{Differential Evolution (DE)}, que se trata de um método de otimização de um problema, onde ele tenta melhorar um resultado. O DE tem o intuito de solucionar um problema, onde uma população \textbf{x} cria novas soluções de candidatos, combinando as existentes de acordo com a fórmula, gerando uma melhor pontuação para problemas de otimização.
 o DE funciona da seguinte maneira 

Inicialização > Mutação > Recombinação > Seleção > Mutação > Recombinação > Seleção > Mutação %transformar em uma imagem

 O programa se repete até obter seus melhores resultados.
Portanto, o programa ocorre através de: \textbf{NP} (Number Parents), \textbf{CR E [0,1]} (Crossover Probability) e \textbf{F E [0.2]} (parâmetro do peso diferencial). Configurações típicas como \textbf{F = 0.8} e \textbf{CR =0.9}. Logo a optimização da performance fornece agentes \textbf{X} aleatórios, dos quais rodam em um espaço de busca até seus requisitos serem atendidos. A escolha de três agentes vetoriais são escolhidas. Em seguida escolhemos um índice de dimensão aleatório, computamos tudo é jogamos na fórmula do \textbf{yi = ai + F x (bi - ci)}, sendo \textbf{a, b e c} os vetores escolhidos aleatoriamente levando em consideração de \textbf{ri <CR} ou  \textbf{i = R}. De caso \textbf{f(y) <= f(x)} substituímos o agente \textbf{x} na população pela melhor solução.


\section{Resultados e Discussões}

\section{Conclusão}

\section{Referências}

\end{document}
